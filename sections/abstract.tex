We investigate the use of Bayesian neural networks (BNNs) in modeling large samples of time delay lenses for Hubble constant ($H_0$) determination. When trained on simulated HST-like images of strongly-lensed quasars, BNNs can accurately characterize the individual lens posterior distributions. We then propagate the BNN-inferred posterior into $H_0$ inference assuming simulated time delay measurements from a plausible dedicated monitoring campaign. 
On a simulated dataset of 400 systems, each consisting of an elliptical power-law lens mass distribution, an elliptical Sersic lens light distribution, an elliptical Sersic model quasar host galaxy, and a point-source quasar, the BNN can recover the true lens model parameters with $\sim 1$\% precision at per-pixel RMS levels of $0.02$, representing comparable performance to traditional forward-modeling approaches, but a million times faster. 
Assuming well-measured time delays (predicted during the 2017 Time Delay Challenge) and a reasonable set of priors on the kinematics and the environment of the lens, this translates to a mean precision of $\sim y$\% per lens in the inferred $H_0$, with mean bias less than $\sim z\%$. A simple combination of the independent $H_0$ estimates provides a forecast of $zz \%$ precision on this cosmological parameter.
In particular, we find that the precision on $H_0$ is very sensitive to accurate and precise source position recovery, and discuss how the BNN inference scheme can be modified to accommodate this requirement.
%The simulated dataset is similar to that used in Rung 1 of the Time Delay Lens Modeling Challenge (TDLMC).